\documentclass[11pt]{article}
\usepackage{amsthm,amsfonts}
\usepackage[french]{babel}
\usepackage[latin1]{inputenc}
\usepackage[T1]{fontenc}
\usepackage{pslatex}
%%%%%%%%%%%%%%%%%%%%%%%%%%%%%%%%%%%%%%%%%�
\input wims.sty
%%%%%%%%%%%%%%%%%%%%%%%%%%%%%%%%%%%%%%%%%%
%\wimsinclude{persowims.sty}%pour inclure des fichiers non lus par latex
%%%%%%%%%%%%%%%%%%%%%%%%%%%%%%%%%%%%%%%%%%
\theoremstyle{plain}
\newtheorem{thm}{Th�or�me}[section]
\newtheorem{theo}[thm]{Th�or�me}
\newtheorem{prop}[thm]{Proposition}
%%%%%%%%%%%%%%%%%%%%%%%%%%%%%%%%%%%%%%%%%%
\theoremstyle{definition}
\newtheorem{defn}{D�finition}[section]
\newtheorem{ex}{Exemple}
\newtheorem{exo}{Exercice}
\newtheorem{rem}{Remarque}
%%%%%%%%%%%%%%%%%%%%%%%%%%%%%%%%%%%%%%%%%%
\def\RR{\mathbb R}
\def\NN{\mathbb N}
\def\ZZ{\mathbb Z}
\def\CC{\mathbb C}
%%%%%%%%%%%%%%%%%%%%%%%%%%%%%%%%%%%%%%%%%%
\begin{document}
%% indispensable
\title{Mon document}
\author{Bernadette Perrin-Riou}
\email{bpr@math.u-psud.fr}
\section{utilisation des environnements}
  \begin{thm}
    Voici un th�or�me
  \end{thm}
  \begin{proof}
    Voici la d�monstration.
  \end{proof}
  \begin{defn}
    Voici une d�finition
  \end{defn}
  \begin{rem}
    Voici une remarque
  \end{rem}
  \begin{defn}[titre]
    Voici une d�finition avec titre
  \end{defn}
\section{Insertion d'un exercice}
Pour ins�rer un exercice, prendre les deux premi�res lignes
dans le fichier source d'une feuille d'exercice et les coller
par un \&.
\begin{exo}
Faites l'exercice d'application\index{exercice1}
\exercise{module=H6/analysis/oefcourbe.fr&exo=courb1&scoredelay=&confparm1=A&confparm1=B}{Courbes}
\end{exo}
\section{Insertion d'un exemple}
Un exemple de dessin non vu dans le fichier pdf \index{exemple}
 \begin{wimsonly}
  \begin{wims}
    \def{matrix A = -6,28,21
      4,-15,-12
     -8,a,25}
    \def{text P = pari(charpoly( [\A], x))}
    \def{text color=blue,purple,red,orange,yellow}
    \def{text dessin = xrange -1,4
      yrange -10,10
      hline 0,0,black
      vline 0,0, black
    }
    \def{text liste = 31.8,31.9,32,32.1,32.2}
    \def{integer cnt = items(\liste)}
    \for{i = 1 to \cnt}{
      \def{text p = evalue(\P,a = \liste[\i])}
      \def{text dessin = \dessin
        plot \color[\i], \p
      }
     }
  \end{wims}
  Voici le graphe des polyn�mes caract�ristiques des matrices \([\A]) pour les valeurs
  de \(a) suivantes : \liste.
    <p class="wimscenter"> \draw{300,300}{\dessin} </p>
  Que remarquez-vous ?
\end{wimsonly}


\section{Une animation qui ne doit pas �tre dans latex}
\begin{wimsonly}
\begin{wims}

\def{text liste=1,2,3,4,10,20,24,28,32,36,40}
\def{text listepoint=-1.8,-1.8,-1.8,-1.5,-1.2,-1.2}
\def{integer cnt =items(\liste)}
\def{text h=wims(makelist x for x = 1 to \cnt)}

\comment{L'entier N est le param�tre qui permet "d'avancer" ;
parm1 est une variable qui va �tre transmis � la page suivante. Au
d�part parm1 ne vaut rien donc N = 1 ; ensuite parm1 vaut N
donc N va s'incr�menter
}
\def{integer N = \parm1 notitemof \h ? 1 : \parm1 + 1}
\def{text commentaire =\N >= \cnt ? Recommencer:Continuer}
\comment{on fait un lien sur la m�me page mais en transmettant
le param�tre \N}
\link{.}{\commentaire}{}{parm1=\N}{dessin}
\comment{Ensuite, c'est du code pour faire un dessin}
\def{text couleur=wims(makelist yellow,orange,pink,blue,green for x = 2 to 5)}
\def{text dessin= xrange -2,2
  yrange -1,1
  hline 0,0,black
  arrow  0,0,1,0,6,black
  arrow 0,0,0,1,6,black
  vline 0,0,black}
\comment{On ne trace que les N premi�res courbes}
\for{i = 1 to \N}{
  \def{integer j = 2*\liste[\i]}
  \def{function f=x/(1+x^(\j))}
  \def{real a=evalue(\f,x=\listepoint[\i])}
  \def{text dessin=\dessin
     plot \couleur[\i],\f
     text \couleur[\i],\listepoint[\i],\a,medium,f\j}
}
<a id="dessin"></a>
\draw{500,300}{\dessin}

\end{wims}
\end{wimsonly}
\section{Les listes}
  Voici une liste avec pli (fold)
  \begin{description}
    \item[Graphe] voici la d�finition
    \item[Coloriage] Colorier un graphe, c'est ...
  \end{description}
\section{Les tableaux}
 \begin{tabular}{|c|c|c|}
 \hline
 &a&\(\frac{1}{2}\) \\
  \hline
 c&d&f\\
  \hline
 \end{tabular}

 \section{Un algorithme}
Vous pouvez configurer l'environnement algorithm dans latex pour
que les mots cl�s soient en fran�ais.

\begin{algorithm}
\caption{Mon algorithme}
\begin{algorithmic}[1]
\REQUIRE $a, b$
\ENSURE  $u$ et $v$ tels que $u a + v b = pgcd(a,b)$
\IF{$a = 0$}
  \STATE Retourner $\lbrack b,0,12 \rbrack$
\ENDIF
  \STATE $x \rightarrow a$,  $y \rightarrow b$, $ vx \rightarrow 0$,  $vy \rightarrow 1$
  \WHILE{ $ y <> 0$}
   \STATE $q \rightarrow \; quotient \; de  x \; par \;  y$, $r \rightarrow x-q*y$
  \STATE $t \rightarrow vy$
  \STATE $vy \rightarrow vx - q*vy$
  \STATE $vx \rightarrow t$
  \STATE $x \rightarrow y$
  \STATE $y \rightarrow r$
\ENDWHILE
 \STATE Retourner $\lbrack x, (x-b*vx)/a, vx \rbrack $
\end{algorithmic}
\end{algorithm}

\end{document}
