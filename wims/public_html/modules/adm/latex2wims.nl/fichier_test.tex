\documentclass[11pt]{article}
\usepackage{amsthm,amsfonts}
\usepackage[french]{babel}
\usepackage[latin1]{inputenc}
\usepackage[T1]{fontenc}
\usepackage{pslatex}
%%%%%%%%%%%%%%%%%%%%%%%%%%%%%%%%%%%%%%%%%�
\usepackage{hyperref}
\usepackage{url}
\usepackage{comment}
\excludecomment{wimsonly}
\includecomment{latexonly}
\usepackage{verbatim}
%%%%%%%%%%%%%%%%%%%%%%%%%%%%%%%%%%%%%%%%%%
\def\wimsentre#1{}
\def\wimsinclude#1{}
% permet le contr�le de la navigation dans le document :%le premier argument peut �tre prev, next,upbl,titb,keyw,datm
\def\wimsnavig#1#2{}
%%%%%% avec le package hyperref, fait apparaitre les liens sur les exercices � 
%partir du fichier pdf. changer l'adresse
\def\exercise#1#2{\href{http://127.0.0.1/wims/wims.cgi?#1&cmd=new}{WIMS : #2}}
\def\doc#1#2{\href{http://wims.auto.u-psud.fr/wims/wims.cgi?#1&cmd=new}{WIMS : #2}}
%%%si n�cessaire
\def\email#1{}
%%%%%%%%%%%%%%%%%%%%%%%%%%%%%%%%%%%%%%%%%%
\wimsinclude{wims.sty}%pour inclure des fichiers non lus par latex
%%%%%%%%%%%%%%%%%%%%%%%%%%%%%%%%%%%%%%%%%%
%%ou
%\input wims.sty
%%%%%%%%%%%%%%%%%%%%%%%%%%%%%%%%%%%%%%%%%%
\theoremstyle{plain}
\newtheorem{thm}{Theorema}[section]
\newtheorem{theo}[thm]{Theorema}
\newtheorem{prop}[thm]{Propositie}
%%%%%%%%%%%%%%%%%%%%%%%%%%%%%%%%%%%%%%%%%%
\theoremstyle{definitie}
\newtheorem{defn}{Definitie}[section] 
\newtheorem{ex}{Voorbeeld}
\newtheorem{exo}{Oefening}
%%%%%%%%%%%%%%%%%%%%%%%%%%%%%%%%%%%%%%%%%%
\def\RR{\mathbb R}
\def\NN{\mathbb N}
\def\ZZ{\mathbb Z}
\def\CC{\mathbb C}
%%%%%%%%%%%%%%%%%%%%%%%%%%%%%%%%%%%%%%%%%%
\begin{document}
%% indispensable
\title{Mijn document}
\author{Bernadette Perrin-Riou}
\email{bpr@math.u-psud.fr}
\section{gebruik van \LaTeX secties}
  \begin{thm}
    Hier een theorema
  \end{thm}
  \begin{proof}
    Hier een bewijs
  \end{proof}
  \begin{defn} 
    Hier een definitie
  \end{defn}
\section{Invoegen van een oefening}

Om een oefening in te voegen, begin u de eerste twee regels
van het bronbestand een oefening met een \&.
\begin{exo}
Maak een oefening van \index{exercice1}
\exercise{module=H6/analysis/oefcourbe.fr&exo=courb1&scoredelay=&confparm1=A&confparm1=B}{Curves}
\end{exo}
\section{Invoegen van een plaatje}

Een voorbeeld van een plaatje
 \index{exemple}
 \begin{wimsonly}
  \begin{wims}
    \def{matrix A = -6,28,21
      4,-15,-12
     -8,a,25}
    \def{text P = pari(charpoly( [\A], x))}
    \def{text color=blue,purple,red,orange,yellow}
    \def{text dessin = xrange -1,4
      yrange -10,10
      hline 0,0,black
      vline 0,0, black
    }
    \def{text liste = 31.8,31.9,32,32.1,32.2}
    \def{integer cnt = items(\liste)}
    \for{i = 1 to \cnt}{
      \def{text p = evalue(\P,a = \liste[\i])}
      \def{text dessin = \dessin
        plot \color[\i], \p
      }
     }
  \end{wims}
  Hier is de grafiek van de polynoom gekarakteriseerd bij de matrices \([\A]) 
  voor de waarden van \(a): \liste.
    <p align="center"> \draw{300,300}{\dessin} </p>
  Wat valt op ? 
\end{wimsonly}
\section{De lijsten}
  Hier een geneste lijst [fold]
  \begin{description}
    \item[Grafiek] Hier is de definitie
    \item[KLeuren] De kleur van de grafiek ...
  \end{description}
\end{document}